\documentclass[]{beamer}
\usetheme{Madrid}
%\setlength{\oddsidemargin}{0.25in}
%\setlength{\evensidemargin}{0.15in}
%\setlength{\topmargin}{0.3in}
%\setlength{\textwidth}{6.0in}
%\setlength{\textheight}{8.5in}
\usepackage{amsmath,amssymb, amsthm}
\usepackage{graphics}
%\usepackage{graphicx}
\usepackage{lscape,graphicx}
\usepackage{amsfonts}
\usepackage{longtable}
\usepackage{epsfig}
\usepackage{float}
\usepackage{rotating}
\usepackage{fancyhdr}
\usepackage[english]{babel}
\usepackage{physics}
\usepackage{hyperref}
\usepackage{bm}
\usepackage{color}
\usepackage{enumerate}
\usepackage{inputenc}
\usepackage{caption}
\usepackage{subfigure}
\usepackage{booktabs}
\usepackage{lipsum}
\usepackage{geometry}
%\usepackage{titlesec}
\usepackage{lipsum}
\usepackage{siunitx}

%\geometry{top = 2cm, bottom = 2cm, left = 2 cm, right = 2cm}

\newcommand{\I}{\mathrm{i}}
\newcommand{\matmel}[3]{\left[ #1 \right]_{#2, #3}}
\newcommand{\complete}[1]{\sum_{#1}\ketbra{#1}{#1}}
\newcommand{\ahat}{\hat{a}}
\newcommand{\Ahatd}{\hat{A}^{\dagger}_{}}
\newcommand{\Ahat}{\hat{A}}
\newcommand{\ahatd}{\hat{a}^{\dagger}}
\newcommand{\uhat}{\hat{u}}
\newcommand{\uhatd}{\hat{u}^{\dagger}_{}}
\newcommand{\phat}{\hat{P}}
\newcommand{\phatd}{\hat{P}^{\dagger}_{}}
\newcommand{\nhat}{\hat{n}}
\newcommand{\psib}{\tilde{\psi}}
\newcommand{\ham}{\hat{H}}
\newcommand{\slims}[2]{\sum\limits_{#1}^{#2}}
\newcommand{\ilims}[2]{\int\limits_{#1}^{#2}}
\newcommand{\cE}{\mathcal{E}}
\newcommand{\cN}{\mathcal{N}}
\newcommand{\cM}{\mathcal{M}}

%opening
\title[]{Monte Carlo Simulation of Spin-fermion Model}
\subtitle{\emph{Based on}\\
Phase Separation in Electronic Models for Manganites\\
S. Yunoki, J. Hu, A. L. Malvezzi,\\ A. Moreo, N. Furukawa, and
E. Dagotto.\\
\emph{Phys. Rev. Lett., 80:845–848, Jan 1998.}
}
\author[A B Sanyal]{Amit Bikram Sanyal}
\date{\today}

\begin{document}
\maketitle



\begin{frame}{Modelling a crystal: Hubbard Model}
\begin{itemize}
	\item Let us consider a model of a rectangular lattice where the electrons can hop to nearest neighboring sites, and interact on the same site. This can be represented by
	\begin{align}\label{eqn:tight-binding-hamil}
	\ham = -t \left(\sum_{\sigma = \uparrow, \downarrow} \sum_{\expval{i,j}}^{} \hat{c}^{\dagger}_{i, \sigma} \hat{c}^{}_{j, \sigma} + \mathrm{h.c.} \right) + U \sum_{i=1}^{N}\hat{n}_{i,\uparrow} \hat{n}_{i,\downarrow}
	\end{align}
	\item This is known as the Hubbard Model.
	\item It is intrinsically a many-body system due to the 4-fermionic interaction term $ \nhat_{i, \uparrow} \nhat_{i, \downarrow} $.
	\item As a result, the Hilbert space grows as $ \order{N!} $: this is an obvious problem for computation.
\end{itemize}
\end{frame}

\begin{frame}{Simplifying to a smaller Hilbert space}
\begin{itemize}
	\item There are several, almost equivalent ways of tacking this problem.
	\item One of these methods is the Mean-field solution, where we assume that a single electron is moving in a a effective field created by the rest of the electrons.
	\item We can show that interaction can be approximately decomposed into \emph{auxiliary fields}, which act as an effective classical spin sitting at each location that interacts with the itinerant electron.
\end{itemize}
\end{frame}

\begin{frame}{The Spin-fermion Model}
\begin{itemize}
	\item The Spin-fermion (SF) model was initially a phenomenological model.
	\item We pretend that each site is actually occupied by a classical spin that can take any orientation, and at each site, a single fermionic interaction term appears that can simulate the effects of the electron-electron interaction.
	\item For an $ L_x \cross L_y $ lattice, this only creates a much smaller Hilbert space of dimension $ 2 L_x L_y $, as opposed to $ \order{(L_x L_y)!} $ in the many-body case. The factor of $ 2 $ appears because of the two spin sectors of the problem.
	\item Although phenomenological in nature, it was shown to be highly effective in capturing the effects of interaction in the system.
\end{itemize}
\end{frame}

\begin{frame}{The Spin-fermion Model}
\begin{itemize}
	\item Later on, it was discovered that there are actual materials in nature with multi-orbital crystals, such as cuprates and manganites, which realistically have a structure similar to the SF model!\cite{dagotto_2011}.
	\item This further cemented the importance of this model as an investigative tool.
\end{itemize}
\end{frame}

\begin{frame}{The Spin-fermion Model}
\begin{itemize}
	\item The single-orbital SF Hamiltonian is given as
	\begin{align}
	\nonumber
	\ham =& -t \slims{\expval{i,j},\sigma}{} \left( {\ahatd}_{i,\sigma} {\ahat}_{j,\sigma} + h.c. \right) - J_H \slims{i}{} \vec{s}_i \cdot \vec{S}_i + J_{AF} \slims{\expval{i,j}}{} \vec{S}_i \cdot \vec{S}_j\\
	\nonumber
	=& -t \slims{\expval{i,j},\sigma}{} \left( {\ahatd}_{i,\sigma} {\ahat}_{j,\sigma} + h.c. \right) - \frac{J_H}{2} \slims{i, \alpha, \beta}{} \ahatd_{i,\alpha} \vec{\sigma}_{\alpha, \beta} \ahat_{i, \beta}  \cdot \vec{S}_i \\
	&+ J_{AF} \slims{\expval{i,j}}{} \vec{S}_i \cdot \vec{S}_j
	\end{align}
	\item For our problem, we will set $ J_{AF} = 0 $.
	\item $ \abs{\vec{S}_i} = 1 \implies \vec{S}_i = \left[\sin(\theta_i) \cos(\phi_i), \sin(\theta_i) \sin(\phi_i), \cos(\theta_i) \right] $
\end{itemize}
\end{frame}

\begin{frame}{Writing the Hamiltonian matrix}
\begin{itemize}
	\item The Hamiltonian matrix can now be evaluated in the single particle basis, which gives the following non-zero elements of the tight-binding Hamiltonian as well as the Hund-interaction Hamiltonian.
	\item For the Hund interaction Hamiltonian, we get the following non-zero terms:
	\begin{subequations}
	\begin{align}
	H_{i,i} &= \frac{J_H}{2} \cos(\theta_i)\\
	H_{i + L_x L_y,i + L_x L_y} &= -\frac{J_H}{2} \cos(\theta_i)\\
	H_{i, i+ L_x L_y} &= \frac{J_H}{2} \sin(\theta_i) \exp(- \I \phi)
	\end{align}
	\end{subequations}
\item $ H = H_{TB} + H_{Hund} $
\item The eigenvalues therefore only depend on the set of $ \left \{ \theta_i, \phi_i \right \} $ at a given temperature and filling.
\end{itemize}
\end{frame}

\begin{frame}{Finding the ground state via Metropolis algorithm}
\begin{itemize}
	\item For sake of simplicity, we fix the classical spins on plane, implying all $ \phi_i=0 $.
	\item The algorithm is now as follows:
	\begin{enumerate}
		\item Start with a random set of $ \theta_i $
		\item Start at site one, and calculate the energy of the system.
		\item Change the value of $ \theta_i $ at that site, and diagonalize the Hamiltonian again to recalculate the energy of that configuration.
		\item Calculate the probability of transition from the old state to the new one.
		\item Roll a random number to probabilistically accept or reject this transition.
		\item Move on to subsequent sites till last site.
		\item Repeat entire process several times till energy values have settled.
	\end{enumerate}
\end{itemize}
\end{frame}

\begin{frame}{Calculation of probability}
\begin{itemize}
	\item We define a function
	\begin{align}
	\nonumber
	P &= \tr(Z)\\
	&= \prod\limits_{\lambda}\left[1 + \exp(-\beta \left( \epsilon_\lambda - \mu \right))\right]
	\end{align}
	\item We calculate $ P' $ after perturbing the angle at a given site. This changes $ P $ as changing the angle changes the Hamiltonian and hence, the spectrum.
	\item Probability of this change being accepted is $ \frac{P}{P'} $.
\end{itemize}
\end{frame}

\begin{frame}{Results}

A phase diagram is taken from \cite{PhysRevLett.80.845}:
	\begin{figure}[h!]
		\centering
		\includegraphics[width=0.4\linewidth]{phase}
		\caption{Phase diagram between $ J_H $ and $ \mu $}
		\label{fig:phase}
	\end{figure}
	
\end{frame}

\begin{frame}{Results}
\begin{figure}[h!]
	\centering
	\includegraphics[width=0.85\linewidth]{spins}
	\caption{Result from my simulation, at half-filling, very low temperature, at $ J_H = 1.0 $.}
	\label{fig:spins}
\end{figure}

\end{frame}


\begin{frame}{References}
\nocite{*}
\bibliography{mybib}{}
\bibliographystyle{acm}
\end{frame}
\end{document}
